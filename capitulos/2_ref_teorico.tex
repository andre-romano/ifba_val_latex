
%% REFERENCIAL TEORICO

\xchapter{\MakeTextUppercase{Revis\~{a}o Bibliogr\'{a}fica}}{}

Uma revisão bibliográfica de TCC deve incluir os seguintes elementos, geralmente escritos como texto corrido (orações e paragrafos), evitando itemizações ou enumerações:

\begin{itemize}
    \item \textbf{Fundamentação Teórica}: Apresentação das principais teorias, conceitos e autores relevantes para o tema.
    \item \textbf{Análise Crítica}: Discussão das contribuições e limitações das obras consultadas, evidenciando lacunas e pontos de divergência na literatura.
    \item \textbf{Evolução do Tema}: Descrição do desenvolvimento histórico e das tendências recentes na área de estudo.
    \item \textbf{Relacionamento com a Pesquisa}: Explicação de como os estudos revisados se conectam com a pesquisa realizada, justificando a escolha do referencial teórico.
    \item \textbf{Síntese}: Resumo dos principais insights e direcionamentos que a revisão proporciona ao trabalho.
\end{itemize}

\textbf{IMPORTANTE}: Tudo que for mencionado na revisão bibliografica, que não tenha sido fruto deste trabalho de TCC, \textbf{deve estar devidamente citado}, a fim de evitar plágio e autoplágio (\url{https://www2.ufjf.br/ppgedumat/discentes/plagio-e-autoplagio/}).