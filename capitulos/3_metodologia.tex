
\xchapter{\MakeTextUppercase{Metodologia}}{} %sem preambulo

A metodologia descreve o caminho que culminou nos resultados encontrados no TCC, permitindo assim a replicação do estudo por outro pesquisador. Desta forma, a metodologia de um TCC deve incluir os seguintes elementos, geralmente escritos na forma de texto corrido (orações e paragrafos), evitando itemizações ou enumerações:

\begin{itemize}
    \item \textbf{Tipo de Pesquisa}: Definição do tipo de estudo (quantitativo, qualitativo, exploratório, descritivo, etc.).
    \item \textbf{Objeto de estudo}: Descrição do fenômeno, evento, processo, grupo, software ou sistema que está sendo investigado.
    \item \textbf{Procedimentos de Coleta de Dados}: Descrição das técnicas e instrumentos utilizados para coleta de dados (questionários, entrevistas, experimentos, etc.).
    \item \textbf{População e Amostra}: Explicação do universo de estudo e dos critérios de seleção da amostra.
    \item \textbf{Análise de Dados}: Métodos e técnicas utilizados para analisar os dados coletados (estatística, análise de conteúdo, etc.).
    \item \textbf{Aspectos Éticos}: Considerações sobre a ética na pesquisa, como consentimento informado e confidencialidade.
\end{itemize}