
%% INTRODUCAO

\xchapter{\MakeTextUppercase{Introdu\c{c}\~{a}o}}{}

Uma introdução de TCC deve incluir os seguintes elementos, geralmente escritos na forma de texto corrido (orações e paragrafos), evitando itemizações ou enumerações:

\begin{itemize}
    \item \textbf{Contextualização}: Apresentação do tema e sua relevância no contexto acadêmico ou social.        
    \item \textbf{Problema de Pesquisa}: Definição clara da questão ou problema que o trabalho pretende investigar.
    \item \textbf{Justificativa}: Explicação da importância do estudo e das contribuições esperadas.
    \item \textbf{Metodologia}: Breve descrição dos métodos e abordagens utilizados na pesquisa.    
\end{itemize}

\textbf{SUGESTÃO}: É preferivel elaborar a introdução no final do trabalho, após ter redigido todas os outros capítulos. Dessa forma, você terá uma visão panorâmica do trabalho, o que permite que você construa a sua introdução como um resumo do conteúdo das demais seções.

\section{OBJETIVOS}

\subsection{\MakeTextUppercase{GERAL}}
Descrever, de maneira geral, o que se deseja alcançar com a proposta descrita na monografia. O objetivo geral é geralmente descrito como um paragrafo simples, que começa com um verbo no imperativo. O verbo deve ser passivel de avaliação nos resultados.

\subsection{\MakeTextUppercase{ESPECÍFICOS}}
Descrever, de maneira detalhada, como o objetivo geral será alcançado, na forma de topicos. Cada objetivo deve começar com um verbo no imperativo, que possa ser avaliado nos resultados.

\begin{itemize}
\item objetivo 1
\item objetivo 2
\item objetivo 3
\end{itemize}

\section{ORGANIZAÇÃO DO TRABALHO}

Esta subseção contem uma breve descrição dos capítulos ou seções que compõem o TCC, indicando a organização do conteúdo da monografia.

