
%% DISCUSSAO

\xchapter{\MakeTextUppercase{Discussão de Resultados}}{} %sem preambulo

Este capítulo deve incluir os seguintes elementos, geralmente escritos na forma de texto corrido (orações e paragrafos), evitando itemizações ou enumerações:

\begin{itemize}
    \item \textbf{Apresentação dos Resultados}: Exposição clara dos dados coletados utilizando tabelas, gráficos, e figuras para facilitar a compreensão.
    \item \textbf{Interpretação dos Resultados}: Análise dos dados à luz do referencial teórico e dos objetivos do estudo, destacando os principais achados e quais objetivos foram alcançados.
    \item \textbf{Discussão Comparativa}: Comparação dos resultados obtidos com outros estudos relevantes na literatura, apontando semelhanças, diferenças e possíveis explicações.
    \item \textbf{Implicações dos Resultados}: Reflexão sobre as contribuições teóricas, práticas e/ou sociais dos resultados, indicando sua relevância e aplicação.
\end{itemize}

