
% CONCLUSAO

\xchapter{\MakeTextUppercase{Conclus\~{a}o}}{} %sem preambulo

Este capítulo deve incluir os seguintes elementos, na forma de texto corrido (orações e paragrafos), sem itemizações ou enumerações:

\begin{itemize}
    \item \textbf{Síntese dos Resultados}: Resumo dos principais achados da pesquisa, respondendo diretamente ao problema de pesquisa e aos objetivos propostos.
    \item \textbf{Contribuições}: Indicação das contribuições do estudo para o campo acadêmico, profissional ou social, destacando sua relevância.
    \item \textbf{Limitações do Estudo}: Reconhecimento das limitações enfrentadas durante a pesquisa e como elas podem ter impactado os resultados.
    \item \textbf{Sugestões para Pesquisas Futuras}: Recomendações para estudos futuros que possam aprofundar ou expandir as questões abordadas.
    \item \textbf{Considerações Finais}: Reflexões finais sobre a importância do estudo, ressaltando a originalidade e as principais implicações.
\end{itemize}

\textbf{IMPORTANTE}: A conclusão não deve conter conteúdo novo. Ela deve servir como um compilado geral do trabalho, geralmente escrita na forma de alguns poucos parágrafos.
